% You insert your abstract in the space below.


%This document is a summary of some relevant commands needed to create
%a Master's thesis for the Department of Mathematics and Statistics
%using \LaTeX. Included are examples of equations, figures, tables, and
%theorems. The formats listed in this document have been approved by
%the Department of Mathematical Sciences and the Graduate Division and
%Research.  If you have any difficulties with any of the driver or
%style files, please see your graduate adviser.

Galactic-scale outflows are thought to be the primary mechanism in the removal of cool gas in star-forming galaxies. Presently, the mass and energy of these flows remain poorly constrained. One way to better constrain these parameters is to measure the spatial extent of the outflow; however, measuring the spatial extent of such outflows via spectral methods has been traditionally very difficult due to the faintness of emission lines tracing outflowing material. We present VLT/FORS2 narrowband imaging of 5 star forming galaxies at redshift z=0.67-0.69 in the GOODS-S field as part of an effort to spatially resolve large-scale outflows traced by MgII emission. Previous spectra of this sample have already revealed winds traced by MgII absorption. At our sample redshift, the MgII emission lines fall exactly within the FORS2 HeII+47 and HeII/3000+48 interference filters. The total integration time of 10 hrs obtained in each filter permits the analysis of the flow surface brightness and extent on scales over which MgII is typically detected in absorption ( i.e, projected distances > 100 kpc). Such measurements can provide stronger constraints on the mass and energy of feedback, helping to advance our understanding of the processes regulating galaxy evolution up to z=1. 
